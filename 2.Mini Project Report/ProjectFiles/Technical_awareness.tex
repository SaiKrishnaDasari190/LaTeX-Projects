\documentclass[12pt]{article}

%Adding Image
\usepackage{graphicx}
\graphicspath{{images/}}

%HyperLinks for Table of Contents
\usepackage{hyperref}
\hypersetup{colorlinks = true, citecolor = blue, linkcolor = blue, urlcolor = blue}

%Title
\title{\textbf{Technical Awareness for Showcasing Your \emph{Technical Skills}}\\}



\author{\textbf{GROUP-4}\\Sai Krishna Dasari\\Ramleela Laxmi Gayatri\\Kutcharlapati Sai Mahitha\\Konna Mahalakshmi\\V.B.G.Sai Kalyan}

%Starting of the Document
\begin{document}
\maketitle %Title will go over here

\begin{figure}[h]
\centering
\includegraphics[scale=0.8]{rgukt_logo.jpg}
\end{figure}
\begin{center}
\textbf{\textbf{Guide}}\\
\textbf{Mr.N.Ramesh Babu} M.Tech\\
Assistant Professor\\
Department of \textbf{ECE}\\
\end{center}


\pagebreak
\tableofcontents %for making table of contents
\clearpage

\section{Description}
Technical awareness is the ability to stay up-to-date with the latest advancements in your field, and to understand how these developments could impact your work. Showcasing your technical skills in an interview involves not only demonstrating your knowledge of specific tools or technologies, but also your ability to think critically and adapt to changes in the industry.\\

To demonstrate technical awareness in an interview, it's important to stay informed about the latest trends and developments in the field. This can involve reading industry publications, attending conferences or webinars, and participating in online forums or discussion groups.\\

Additionally, you should be prepared to discuss how you have applied your technical skills in previous roles, and to provide examples of how you have adapted to changes in the industry. This could involve discussing projects you have worked on, or highlighting any training or professional development you have pursued to stay current with new technologies or techniques.\\

Demonstrating technical awareness in an interview can help you stand out as a candidate who is not only knowledgeable, but also proactive and adaptable.

\begin{figure}[h]
\centering
\includegraphics[scale=0.2]{github.png}
\label{github}
\caption{GitHub}
\end{figure}
\begin{figure}[h]
\centering
\includegraphics[scale=0.2]{linkedin.png}
\label{linkedin}
\caption{LinkedIn}
\end{figure}

\section{Showcasing Technical Achievements Online}
It's true that many candidates struggle to effectively showcase their projects and achievements in their resumes and CVs. One way to overcome this challenge is to create repositories and profiles on platforms like GitHub and LinkedIn.


By creating a GitHub repository, you can showcase your coding skills by sharing your projects and contributions to open-source projects. This provides potential employers with a tangible example of your technical abilities and can help demonstrate your passion for your field.

Similarly, LinkedIn profiles can be used to showcase your professional achievements, including any technical projects that you have worked on. This can involve providing a detailed description of the project, as well as any results or metrics that you achieved. Additionally, you can use LinkedIn to showcase any relevant certifications or courses that you have completed to enhance your technical skills.

Overall, creating repositories and profiles on platforms like GitHub [Fig:\ref{github}] and LinkedIn[Fig:\ref{linkedin}] can be an effective way to showcase your technical skills and accomplishments. By providing concrete examples of your work, you can help potential employers understand the value that you could bring to their organization and increase your chances of success in the interview process.

\section{Why GitHub ?} 
Certainly! While there are other platforms that can be used to showcase your technical skills and projects, GitHub is a popular and well-respected platform among technical professionals for a few reasons:[Fig:\ref{github}]

Version control: GitHub uses Git, which is a popular version control system used by developers to manage changes to their code over time. This makes it easy to collaborate on projects with others, and to keep track of changes to your code.

Open-source community: GitHub is known for its large and active community of developers who contribute to open-source projects. By contributing to open-source projects, you can showcase your coding skills and potentially gain recognition from others in your field.

Code hosting: GitHub provides a free and easy way to host your code online, making it easy to share your projects with potential employers or collaborators.

Industry recognition: Many companies and organizations use GitHub to host their own projects and to recruit new talent, which makes it a great platform to showcase your skills to potential employers.

Of course, there are other platforms that can be used to showcase your technical skills, such as GitLab or Bitbucket. However, GitHub is the most widely used and recognized platform in the industry, which is why it's often recommended to technical professionals looking to showcase their skills and projects.
\section{How to Showcase on GitHub?}
When creating and uploading your projects to GitHub, it's important to follow some best practices to ensure that your work is easy to understand and use for others. Here are some key patterns you should follow:\\

\textbf{Organize your code:} Use clear and concise names for your files and directories, and create a logical directory structure to make it easy for others to navigate your code. Use comments to explain what your code does and how it works.\\

\textbf{Use a consistent style:} Follow a consistent coding style throughout your project, including indentation, spacing, and naming conventions. This makes your code more readable and easier to understand.\\

\textbf{Use version control:} Use Git to manage changes to your code over time, and make frequent commits to keep a detailed history of your work.\\

\textbf{Include a README:} Write a clear and concise README file that explains what your project does, how to install and use it, and any other relevant information that others might need to know.\\

\textbf{Use descriptive commit messages:} Write clear and descriptive commit messages that explain the changes you made and why you made them.\\

By following these patterns, you can create high-quality projects on GitHub that are easy to understand and use for others. This can help you gain recognition from others in your field, and can demonstrate your technical skills to potential employers.

\subsection{Key Components for Strong Technical Portfolios}
including documentation, firmware, and hardware folders in your GitHub repository can be a good way to make it easy for interviewers to review your technical projects.

\subsubsection{Documentation:} Include a folder for documentation that describes what your project does, how it works, and any instructions on how to use it. This can be in the form of a README file or additional documentation files in a separate folder. Make sure that your documentation is clear and well-organized, so that anyone reviewing your project can easily understand what you've built.

\subsubsection{Firmware:} If your project involves firmware (e.g. for an embedded system), include a folder that contains the source code for your firmware. Make sure that your code is well-organized and includes comments to explain what each part of the code does.

\subsubsection{Hardware:} If your project involves any hardware components, include a folder that contains any schematics, diagrams, or other files related to the hardware. This can help reviewers understand how the hardware components fit together and how they interact with the firmware.\\

By including these components in your GitHub repository, you can make it easy for interviewers to review your project and gain a better understanding of your technical skills. It also demonstrates your attention to detail and commitment to creating high-quality projects.

\end{document}